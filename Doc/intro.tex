%	INTRODUZIONE


\subsection{Descrizione del Gioco}
	
	Touch the color, o "Strega chiama color" nella sua versione italiana, � un semplice gioco di gruppo da svolgersi all'aria aperta. Questo prevede la selezione di un giocatore del gruppo al ruolo di strega, il quale dovr� scegliere un colore ed urlarne il nome a tutti gli altri componenti del gruppo. Quest'ultimi, udito il colore, dovranno cercare nel minor tempo possibile un oggetto del colore indicatogli per poi raggiungerlo: perde l'ultimo giocatore a "toccare il colore" e viene selezionato come strega al prossimo giro. 
	
	Nel resto della documentazione i robot nel ruolo della strega verranno chiamati nodi Witch e quelli nel ruolo del resto del gruppo verranno chiamati nodi Kid.
	
	\subsection{Nicchia Ecologica}
	La nicchia ecologica permette di definire vincoli e potenzialit� del sistema.
	L'ambiente nel quale i robot operano deve presentare le seguenti caratteristiche:
	\begin{itemize}
		\item pavimentazione omogenea e regolare;
		\item illuminazione forte e omogenea;
		\item oggetti pieni, fonoriflettenti ed aventi altezza almeno pari a quella dei sonar;
		\item strettoie aventi larghezza almeno 1m;
		\item ambiente leggermente dinamico.
	\end{itemize}
	
	Per quanto riguarda il task, abbiamo delle differenze fra nodi Witch e nodi Kid. Per i primi si tratta di un task molto semplice, esso prevede nell'ordine:
	\begin{itemize}
		\item la scelta di un colore da un pool ben definito;
		\item la comunicazione del colore ai nodi Kid;
		\item la comunicazione della fine del gioco ai nodi Kid una volta ricevuto $n-1$ messaggi di "colore toccato", dove $n$ � il numero di nodi Kid.
	\end{itemize}
	Si tratta quindi, per i nodi Witch, di un task a tempo illimitato, robot-based, senza movimento e dipendente, in quanto la comunicazione della fine del gioco dipende da comunicazioni precedenti dei nodi Kid.
	
	\newpage
	Per i nodi Kid, invece, il task prevede:
	\begin{itemize}
		\item trovare un oggetto del colore comunicatogli e raggiungerlo;
		\item comunicare al nodo Witch che ha toccato il colore.
	\end{itemize}
	I nodi Kid quindi svolgono un task minimum-time, in quanto devono cercare di portare a termine il task nel minor tempo possibile, object-based, movement-to e dipendente, per motivazioni simili a quelle fatte per i nodi Witch.
	
	Per quanto riguarda i robot, si tratta di \textit{Pioneer} modello 3-DX munito di due ruote motrici fisse ed una mobile e sei (o dodici) sensori sonar. Sulla base superiore � stata installata una telecamera RGB-D (\textit{Microsoft Kinect}) la quale viene utilizzata per le funzioni di computer vision.
	
	Per quanto riguarda l'organizzazione dei robot, invece, abbiamo una composizione eterogenea, dalla grandezza limitata, la cui riconfigurabilit� � comunicata, avente comunicazione esplicita con topologia ad albero.
	
\subsection{Schema dei Behaviours}

	Come per il resto, lo schema dei behaviours si differenzia in base al tipo di nodo che consideriamo. La Figura~\ref{fig:stregabeh} mostra lo schema riferito ai nodi Witch, la Figura~\ref{fig:bambinobeh} quello riferito ai nodi Kid.
	
	% Figura 
	\begin{figure}[h!]
		\centering
		\includegraphics[width=0.9\textwidth]{images/stregabeh.pdf}
		\caption{Schema dei behaviours dei nodi Witch.}
		\label{fig:stregabeh}
	\end{figure}
	
	% Figura 
	\begin{figure}[h!]
		\centering
		\includegraphics[width=0.9\textwidth]{images/bambinobeh.pdf}
		\caption{Schema dei behaviours dei nodi Kid.}
		\label{fig:bambinobeh}
	\end{figure}
