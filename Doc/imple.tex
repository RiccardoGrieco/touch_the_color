%	DETTAGLI IMPLEMENTATIVI

\subsection{Role Manager}
	La gestione dei ruoli avviene in modo dinamico: ogni volta che inizia un nuovo turno, il robot sa quale roulo dovr� ricoprire. Tale gestione � demandata interamente alla classe \texttt{RoleManager}, la quale � responsabile di:
	\begin{itemize}
		\item settaggio parametri;
		\item lancio dei nodi Kid e Witch;
		\item creazione e connessione delle socket;
		\item gestione dei messaggi.
	\end{itemize}
	Al primo avvio, il sistema impone che il primo robot a ricoprire il ruolo di Witch sia quello con indirizzo IP minore rispetto a tutti gli altri. Per tutti gli altri match, il Role Manager del robot perdente lancia il nodo Witch e tutti gli altri quello Kid.

\subsection{Problemi Relativi alla Sensoristica}
	Durante la fase di test sono state individuate due principali problematiche relative alla sensoristica dei robot: una riguardante i sonar e l'altra la telecamera.
	Per quanto riguarda i sonar, sono stati riscontrati i classici problemi di riflessione delle onde a ultrasuoni. Infatti, utilizzando uno scatolo quadrato come ostacolo, nel momento in cui il robot di muove in direzione di un suo spigolo, i sonar non riescono ad individuare l'ostacolo in quanto le onde emesse vengono riflesse altrove, non ritornando al sensore. Per quanto riguarda invece il problema del \textit{cross-talk}, � stato riscontrato ma raramente.
	
	Le problematiche relative alla telecamera RGB-D riguardano i valori di profondit� per oggetti a distanza minore di $\simeq 0.5$ e il tempo di elaborazione dei frames. Per tali oggetti infatti la distanza calcolata dalla telecamera risulta avere valore \texttt{NaN}. Il problema � stato gestito utilizzando una stima della posizione del POI precedentemente rilevata: quando la misurazione � \texttt{NaN}, ma il POI si trova ad una distanza stimata limite, l'oggetto viene riconosciuto come "toccato".
	Al fine di rendere il movimento del robot pi� fluido, si � deciso di richiamare la funzione di elaborazione dell'immagine con cadenza regolare poich� l'elevato tempo di elaborazione diminuiva la frequenza di esecuzione del loop di controllo. Ci� causava movimenti troppo lunghi che, in situazioni critiche, a condizioni di stallo.